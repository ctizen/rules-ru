\section{Приложения}

\subsection{Краткие отличия от правил ЕМА}

Данные правила основаны на правилах Европейской ассоциации маджонга, со следующими отличиями:

\begin{itemize}
	\item Акадоры: есть;
	\item Абортивные ничьи: есть, кроме абортивной ничьей на тройном роне;
	\item Счетный якуман: есть;
	\item Сложение якуманов: есть.
\end{itemize}

В правилах также могут присутствовать менее значимые отличия, пожалуйста, обращайтесь к полному тексту правил для уточнения.

\newpage

\subsection{Требования к турнирам для аккредитации}

Для аккредитации турнира в рейтинге RR, турнир должен соответствовать следующим требованиям:

\begin{itemize}
	\item Количество игроков: \textbf{16 и более}.
	\item Количество игр: \textbf{4 и более}.
	\item Турнир должен быть \textbf{открытым}.
	\item Правила и схема рассадок на турнире не должны существенно отклоняться от рекомендаций, описанных в данном регламенте. Список допустимых вариаций правил приведен ниже.
\end{itemize}

Открытый турнир --- турнир, в котором как минимум половина мест доступна для открытой регистрации (доступна любому игроку), либо как минимум половина мест распределяется на основании результатов отборочного тура (квалификации), проводимого перед турниром. Квалификация может быть заменой открытой регистрации. Регистрация на квалификацию должна быть открытой для любого игрока.

Допустимые \textbf{вариации правил} (варианты, соответствующие текущему регламенту, выделены жирным; предпочитайте их по возможности):
\begin{itemize}
	\item Акадоры: \textbf{есть} / нет;
	\item Стартовые очки: 25000 / \textbf{30000};
	\item Ума: предпочтительно \textbf{15/5} либо 30/10. Допускается любая другая симметричная ума.
	\item Нагаши манган: есть / \textbf{нет};
	\item Банкротство (досрочное прекращение игры при уходе игрока в минус): есть / \textbf{нет};
	\item Абортивные ничьи: \textbf{есть, кроме тройного рона} / есть, все / нет;
	\item Атамаханэ: есть / \textbf{нет};
	\item Чомбо: \textbf{-20000 после умы} / обратный манган;
	\item Якитори: есть / \textbf{нет};
	\item Ренхо: \textbf{манган} / отсутствует.
\end{itemize}

Необычные \textbf{вариации рассадок}, используемые на турнире, могут также послужить поводом для недопуска турнира в рейтинг RR. Мы рекомендуем придерживаться следующих вариантов:
\begin{itemize}
	\item Рассадка на турнире полностью предопределенная. Организаторы заранее готовят рассадку на весь турнир и по ней происходят игры. Предопределенная рассадка позволяет глобально и заранее минимизировать пересечения между игроками, но требует предварительной работы от организаторов по ее формированию согласно количеству игроков.
	\item Рассадка на турнире полностью автоматическая и состоит из следующих этапов последовательно:
	\begin{itemize}
		\item 1 или 2 случайных рассадки подряд. На турнирах среднего размера стоит ограничиться одной случайной рассадкой в самом начале; на больших турнирах допускается играть по случайной рассадке первые две игры;
		\item Швейцарская рассадка на все игры в середине турнира;
		\item 2 или 3 интервальные рассадки в конце турнира. Выбор интервала оставляется на усмотрение организаторов. Для мелких однодневных турниров стоит ограничиться одной интервальной рассадкой в последней игре.
	\end{itemize}
\end{itemize}

\newpage

\section{Заключение}

В данном регламенте мы постарались максимально охватить как все вопросы, регулярно возникающие у начинающих игроков, так и нюансы в процессе проведения турниров и клубных игр. В случае возникновения каких-либо вопросов или уточнений, просим обращаться в рабочую группу по правилам, все вопросы обязательно будут рассмотрены и при необходимости будут внесены правки.

Редакции правил предполагается обновлять раз в год. В случае отсутствия существенных правок, выпуск новой редакции может быть отменен. При проведении турниров просим указывать версию (год выпуска) регламента правил, которые вы собираетесь использовать на турнире. Ожидается, что все судьи турнира ориентируются в регламенте и следят за актуальностью своих знаний. Рабочая группа обязуется обеспечить максимальное публичное распространение новых редакций по мере их выхода.

Спасибо за внимание и ждем вас на турнирах!
