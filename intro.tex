\section{Введение}

Приветствуем всех интересующихся риичи-маджонгом, опытных игроков и желающих освоить тонкости проведения турниров и судейства. В этом руководстве мы рассмотрим все эти аспекты, дадим строгие определения и игровые правила, рекомендуемые для использования в турнирах по риичи-маджонгу.

Выражаем благодарность всем, кто участвовал в составлении и корректировке правил, всем клубам, предоставившим конструктивную критику и всем игрокам, использующим данный свод правил в качестве основного.

\newpage

\section{Краткие отличия от правил ЕМА}

Данные правила основаны на правилах Европейской ассоциации маджонга, со следующими отличиями:

\begin{itemize}
	\item Акадоры: есть;
	\item Абортивные ничьи: есть, кроме абортивной ничьей на тройном роне;
	\item Счетный якуман: есть;
	\item Сложение якуманов: есть.
\end{itemize}

В правилах также могут присутствовать менее значимые отличия, пожалуйста, обращайтесь к полному тексту правил для уточнения.