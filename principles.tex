\section{Общие принципы}

В этом разделе мы рассмотрим общие принципы, по которым формулируются правила игры, требования к игрокам, правила этикета и иные пункты регламенты.

Турнирные правила формулируются \textit{разумным образом} --- это означает, что подавляющее большинство игроков не имеют вопросов к конкретным формулировкам и согласны с ними. В случае, если у игроков возникают массовые вопросы и претензии к конкретным пунктам регламента, это является поводом для вынесения вопроса на общее обсуждение с целью пересмотрения формулировок или правил и соответствующего изменения регламента.

Регламент должен удовлетворять следующим требованиям:
\begin{itemize}
	\item \textbf{Непротиворечивость} --- пункты регламента должны быть логически согласованы между собой и не допускать противоречий в описаниях и трактовках.
	\item \textbf{Однозначность} --- пункты регламента должны быть сформулированы таким образом, чтобы не допускать двусмысленных трактовок игровых ситуаций и возможностей по их разрешению.
	\item \textbf{Полнота} --- пункты регламенты должны описывать максимально возможное количество игровых ситуаций. В случае, если за столами неоднократно возникает ситуация, не описанная в регламенте, регламент следует дополнить описанием данной ситуации.
	\item \textbf{Непредвзятость} --- пункты регламента должны быть сформулированы справедливым образом, не допускающим получение необоснованного преимущества в игре.
\end{itemize}

Разрешение игровых ситуаций должно удовлетворять следующим условиям:
\begin{itemize}
	\item Все игроки за столом должны высказать свое мнение относительно спорной игровой ситуации. Нельзя допускать молчаливого согласия или давления авторитетов.
	\item Если из-за неточности (в механике, в действиях игрока) возникает возможность неоднозначной интерпретации ситуации, такую неточность следует пресекать.
	\item Разрешение спорных ситуаций должно соответствовать принципам честной игры (fair play), то есть не допускать безосновательных перекосов в пользу того или иного игрока.
\end{itemize}

Одним из дополнительных требований к регламенту является его \textbf{интуитивность}, т.е. соответствие правил тому, к чему привыкло подавляющее число игроков. Это требование является основной причиной для того, чтобы принять правила Европейской ассоциации как базу для данного регламента. Со временем привычки игроков могут меняться, в этом случае следует отразить изменившиеся привычки в регламенте (принцип \textit{дескриптивности}, сравните с принципом прескриптивности, когда выполнение правил требуется от игроков без учета того, к чему они привыкли).

\subsection{Изменения в регламенте}

Регламент публикуется \textit{раз в полгода}. Организаторы турниров и судьи обязаны ознакомиться с изменениями в регламенте, если они имели место.

В случае, если требуются изменения в регламенте, определяется следующий порядок внесения изменений:
\begin{itemize}
	\item Определение конкретных пунктов, в которых требуются правки;
	\item Предварительный сбор мнений с игроков. Для рассмотрения предложения требуется, чтобы не менее 10 игроков поддержали предлагаемые изменения.
	\item Вынесение предложения на обсуждение в сообществе. Этот пункт также может предполагать голосование всех заинтересованных лиц. Круг заинтересованных лиц не регламентируется и принимаются доводы от любого игрока.
	\item Внесение изменений в регламент. Публикация регламента осуществляется по указанному выше графику. До публикации изменений следует пользоваться текущей версией регламента, однако допускается точечное внесение грядущих изменений в регламенты турниров организаторами.
\end{itemize}

Каждое изменение в регламенте должно сопровождаться кратким списком того, что именно было изменено, чтобы людям не приходилось перечитывать регламент полностью каждый раз. В списке изменений следует указывать конкретный номер пункта регламента, который был изменен, а также приводить описание изменения.
